\thispagestyle{empty}

\begin{center}
{\large\bfseries DayDay \\ Gestión de citas para clínicas }\\
\end{center}
\begin{center}
Inés Nieto Sánchez\\
\end{center}

%\vspace{0.7cm}

\vspace{0.5cm}
\noindent\textbf{Palabras clave}: \textit{gestión de citas}, \textit{clínica}, \textit{planificación}, \textit{SCRUM}, \textit{calendario}, \textit{agenda}, \textit{despliegue}, \textit{usabilidad}
\vspace{0.7cm}

\noindent\textbf{Resumen}\\
Tras siete años desde la apertura de la clínica Carmen Verdejo, ésta ha decidido que es hora de evolucionar y apostar por plataformas que agilicen esos lentos procesos administrativos, como es la gestión de citas. Es muy común en muchas empresas, y especialmente en las más pequeñas, a ser reacias al cambio, pues no confían en otro tipo de soluciones para sus tareas a parte de las soluciones tradicionales. En este caso, la clínica Carmen Verdejo ha probado algunas herramientas para la gestión de sus citas, tanto generales como específicas. Por un lado, las generales eran demasiado complicadas de adaptar para flujos de usuario tan simples. Por otro lado, los productos software específicos ofrecían demasiadas funcionalidades, eran caros y lo más importante, era muy difícil aprender a usarlos. Así nace este proyecto, que tiene el objetivo de crear una plataforma de gestión de citas para la clínica Carmen Verdejo. En dicha plataforma, con una distinción de roles para la cesión de permisos, se podrán crear, editar y borrar citas agilizando de esta manera la tradicional metodología que se posee actualmente para dichas tareas. 
 
\cleardoublepage

\begin{center}
	{\large\bfseries DayDay \\ Appointment scheduling for clinics}\\
\end{center}
\begin{center}
	Inés Nieto Sánchez\\
\end{center}
\vspace{0.5cm}
\noindent\textbf{Keywords}: \textit{appointment scheduling}, \textit{clinic}, \textit{planning}, \textit{SCRUM}, \textit{calendar}, \textit{agenda}, \textit{deploy}, \textit{usability}
\vspace{0.7cm}

\noindent\textbf{Abstract}\\
After seven years of Carmen Verdejo clinic's opening, their personnel has decided that it is time to progress and to put faith in platforms that can speed up those slow administrative processes, such as clinic appointments. It is common for many companies, and specially the small ones, to be reluctant to change, they don't trust other solutions for their tasks apart from the tranditional ones. In this case, Carmen Verdejo clinic has tested some tools for their appointment scheduling, general as well as specific ones. On the one hand, the general ones were too complicated to adapt for such as simple user flow. On the other hand, specific software products offered too many funcionalities, they were expensive and the most important factor, it was really hard to learn how to use them. That's how this project was born, it aims to create an intuitive platform for scheduling appointments. In this platform, with a role system for permission management, users will be able to create, edit and remove appointments, so current traditional scheduling appointments methodology is sped up for those tasks.

\cleardoublepage

\thispagestyle{empty}

\noindent\rule[-1ex]{\textwidth}{2pt}\\[4.5ex]

Dª. \textbf{Rocío Celeste Romero Zaliz}, Profesora del Departamento de Ciencias de la Computación e Inteligenia Artificial

\vspace{0.5cm}

\textbf{Informo:}

\vspace{0.5cm}

Que el presente trabajo, titulado \textit{\textbf{DayDay. Gestión de citas para clínicas}},
ha sido realizado bajo mi supervisión por \textbf{Inés Nieto Sánchez}, y autorizo la defensa de dicho trabajo ante el tribunal
que corresponda.

\vspace{0.5cm}

Y para que conste, expiden y firman el presente informe en Granada a Noviembre de 2022.

\vspace{1cm}

\textbf{La tutora: }

\vspace{5cm}

\noindent \textbf{Rocío Celeste Romero Zaliz}

\chapter*{Agradecimientos}
En primer lugar quería agradecer a mi madre, mi padre y mi hermano el apoyo incondicional que me han dado durante toda mi trayectoria estudiantil, sobretodo en esta última y más dura etapa. A mi abuela, quien siempre me transmite lo orgullosa que está de lo lejos que he llegado. A Sergio, del cual desde primero de carrera llevo fascinándome y aprendiendo día tras día. A Alberto, por la amistad que hemos forjado durante la carrera que es de las cosas más especiales que me llevo de ella. A Javi, quien un día fue un compañero de grupo aleatorio en clase y ahora gran amigo. \bigskip

Además, quería agradecer de forma general a todo el que ha pasado por mi camino en la carrera y que de una forma u otra me ha aportado algo en mayor o menor medida. \bigskip

Por último quería agradecer enormemente a mi tutora Rocío, quien me ha guiado en esta recta final donde pongo punto y final a cuatro años de carrera. 



